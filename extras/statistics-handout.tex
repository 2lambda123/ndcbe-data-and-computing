\documentclass[12pt,twoside]{article}

\addtolength{\textwidth}{0.5in}
\usepackage{epsfig,amsfonts,color}
%\usepackage[leqno]{amsmath}
\usepackage{amsmath}
%\usepackage{showlabels}
\newcommand{\norm}[1]{\left| \left| #1 \right| \right|}
\newcommand{\mods}[1]{\left|  #1 \right|}
\newcommand{\cm}[1]{{\color{black} #1}}
\newcommand{\cv}[1]{{\color{black} #1}}
\bibliographystyle{plain}
\usepackage{amssymb, palatino, geometry,url}
\usepackage{algorithmic}
\usepackage[noresetcount,lined,boxed]{algorithm2e} % ... for algorithms
\newcommand{\mynote}[1]{{ \color{black}{#1}}}
\usepackage[colorlinks=true,linkcolor=blue,citecolor=blue,urlcolor=blue]{hyperref}

\usepackage{multirow}

\geometry{letterpaper,
          left       = 0.9in,
          right      = 0.9in,
          top        = 0.9in,
          bottom     = 0.9in}
\linespread{1.2}

\usepackage{fancyhdr}
\pagestyle{fancy}

%\usepackage{lineno}
%\linenumbers

% Customize spacing in lists
\usepackage{enumitem}

% Globally set noindent
\setlength{\parindent}{0pt}

% Use to white out portion students will fill in

\newcommand{\fillin}[1]{{\color{red} \large #1}}

% For subfigures
\usepackage{graphicx}
\usepackage{caption}
\usepackage{subcaption}

% For floatbarrier
\usepackage{placeins}

% Check list
\newlist{todolist}{itemize}{2}
\setlist[todolist]{label=$\square$}

% Comments
\usepackage{verbatim}

\usepackage{tcolorbox}

\usepackage{mhchem}

% With subsections to alphabetical labeling
\renewcommand{\thesubsection}{\thesection.\alph{subsection}}

\newcommand{\assignmentname}{Probability, Statistics, \& Error Propagation Summary}

\lhead{\assignmentname}
\rhead{A.~Dowling, U.~Notre Dame}

%\renewcommand\contentsname{Outline for Lecture X}

\title{\assignmentname}
\author{CBE 20258: Numerical and Statistical Analysis}


% define notation for vectors and matrix
% https://tex.stackexchange.com/questions/229543/double-tilde-symbol-under-letter
\usepackage{stackengine}
\stackMath
\newcommand\tenq[2][1]{%
 \def\useanchorwidth{T}%
  \ifnum#1>1%
    \stackunder[0pt]{\tenq[\numexpr#1-1\relax]{#2}}{\scriptscriptstyle\sim}%
  \else%
    \stackunder[1pt]{#2}{\scriptscriptstyle\sim}%
  \fi%
}

% Vector
\newcommand{\vc}[1]{\tenq{#1}}

% Matrix
\newcommand{\mt}[1]{\tenq[2]{#1}}


% Enable solutions
\usepackage{comment}
\specialcomment{sln}{\begingroup\color{red} \vspace{6pt} \textbf{Solution}: \vspace{6pt} \\}{\endgroup%\vfill\pagebreak
}

\newcommand{\ad}{\vspace{6pt}\textbf{Additional Discussion}:\vspace{6pt}}

% Number equations within subsection
% \numberwithin{equation}{section}

% Hide solutions
\excludecomment{sln}

% Expected value
\DeclareRobustCommand{\bbone}{\text{\usefont{U}{bbold}{m}{n}1}}

\DeclareMathOperator{\EX}{\mathbb{E}}% expected value
\DeclareMathOperator{\PR}{\mathbb{P}}% probability
\DeclareMathOperator{\VR}{\mathbb{V}}% variance

\begin{document}

%\tableofcontents

\date{February 12 - 14, 2018}

% \maketitle

%%%%%%%%%%%%%%%%%%

\section{Probability Distributions}

Probability Density Function (PDF):

\begin{equation}
	\PR{}[a \leq X \leq b] = \int_a^{b} f_X(x) dx
\end{equation}

Important properties include $\int_{-\infty}^{\infty} f_X(x) dx = 1$ and $f_{X}(x) \geq 0$. \\

\textbf{Example}: \emph{Uniform Distribution}. $f_X(x) = \frac{1}{b-a}$ for $x \in [a,b]$, $0$ otherwise. \\

Cumulative distribution function (CDF):

\begin{equation}
	\PR{}[X \leq a] = \int_{-\infty}^{a} f_X(x) dx = F_X(a)
\end{equation}


\section{Population Statistics}

Expected Value and Mean (1st moment):

\begin{equation}
\mu_X = \EX[X] = \int_{-\infty}^{\infty} x f_{X}(x) dx
\end{equation}

\textbf{Example}. What is $\EX[X]$ if $X$ is \emph{uniformly distributed} over $[a, b]$?

\vspace{0.15\textheight}

Variance (2nd moment):

\begin{equation}
	\sigma^2_{X} = \VR{}[X] = \EX[(X - \mu_X)^2] = \int_{-\infty}^{\infty} (x - \mu_X)^2 f_X(x) dx
\end{equation}

\textbf{Example}. What is $\VR[X]$ if $X$ is \emph{uniformly distributed} over $[a, b]$?

\vfill

\textbf{Exam Practice}: Using the above definitions, show that $\VR[X] = \EX[X^2] - (\EX[X])^2$.

\pagebreak

\section{Sample Statistics}

We cannot directly measure population distributions; instead we gather samples and \emph{infer} distribution properties. \\

\begin{minipage}{0.48\textwidth}
Sample Mean:
\begin{equation}
	\bar{x} = \frac{1}{N} \sum_{i=1}^{N} x_i
\end{equation}
\end{minipage} \quad
\begin{minipage}{0.48\textwidth}
Sample Variance:
\begin{equation}
	s^2_x = \frac{1}{N-1}\sum_{i=1}^{N}(x_i - \bar{x})^2
\end{equation}
\end{minipage}

\vspace{12pt}

\begin{minipage}{0.48\textwidth}
Sample Standard Deviation:
\begin{equation}
	s_x = \sqrt{s_x^2}
\end{equation}
\end{minipage} \quad
\begin{minipage}{0.48\textwidth}
Sample Covariance:
\begin{equation}
	s_{x,y} = \frac{1}{N-1}\sum_{i=1}^{N}(x_i - \bar{x})(y_i - \bar{y})
\end{equation}
\end{minipage}

\section{Error Propagation}

\emph{Note}: The following formulas are written for random variables and population statistics. The sample formulas also apply for sample statistics. \\

\begin{minipage}{0.45\textwidth}
\textbf{Addition}. Consider $Z = c_1 X + c_2 Y$,
\begin{equation}
	\sigma_Z^2 = c_1^2 \sigma_X^2 + c_2^2 \sigma_Y^2 + 2 c_1 c_2 \sigma_{X,Y}
\end{equation}
\end{minipage} \quad
\begin{minipage}{0.55\textwidth}
\textbf{Subtraction}. Consider $Z = c_1 X - c_2 Y$,

\begin{equation}
	\sigma_Z^2 = c_1^2 \sigma_X^2 + c_2^2 \sigma_Y^2 - 2 c_1 c_2 \sigma_{X,Y}
\end{equation}
\end{minipage}

\vspace{18pt}

\textbf{Multiplication}. Consider $Z = X \cdot Y$,
\begin{equation}
	\sigma_Z^2\approx \left(\EX[Z]\right)^2 \left[ 
	\left(\frac{\sigma_{X}}{\EX[X]} \right)^2 + \left(\frac{\sigma_{Y}}{\EX[Y]} \right)^2 + \frac{2 \sigma_{X,Y}}{\EX[X] \EX[Y]}
	\right]
\end{equation}


\textbf{Division}. Consider $Z = X / Y$,
\begin{equation}
	\sigma_Z^2\approx \left(\EX[Z]\right)^2  \left[ 
	\left(\frac{\sigma_{X}}{\EX[X]} \right)^2 + \left(\frac{\sigma_{Y}}{\EX[Y]} \right)^2 - \frac{2 \sigma_{X,Y}}{\EX[X] \EX[Y]}
	\right]
\end{equation}

\vspace{6pt}

\textbf{Differentiable Function}. Consider $Z = g(X,Y)$, 

\begin{equation}
	\sigma_Z^2 \approx \left| \frac{\partial g}{\partial X} \right|^2 \sigma_X^2 + \left| \frac{\partial g}{\partial Y} \right|^2 \sigma_Y^2 + 2 \frac{\partial g}{\partial X} \frac{\partial g}{\partial Y} \sigma_{X,Y}
\end{equation}

\vspace{12pt}

\textbf{Exam Practice.} Use the definition of mean and variance to prove the addition and subtraction rules, e.g., $\VR[c_1 X+ c_2 Y] = c_1^2 \VR[X] + c_2^2 \VR[Y] + 2 c_1 c_2 \mathrm{cov}(X,Y)$. For simplicity, assume $X$ and $Y$ are linearly independent, i.e., $f_Z(x,y) = f_X(x) f_Y(y)$ and $\mathrm{cov}(X,Y) = 0$.

\end{document}
